\graphicspath{%
{\PATH/\TITLE/fig00/}%
}

\begin{titlepage}
    \sffamily%
    \begin{fullwidth}%
        \fontsize{36}{40}\selectfont\par\noindent\textcolor{darkgray}{\centering%
        \uppercase{Perceptual Assessment \\ of Sound Field Synthesis\\}
        }%
        \vspace{1.6cm}%
        \fontsize{18}{20}\selectfont\par\noindent\textcolor{darkgray}{\centering%
        vorgelegt von\\
        Dipl.-Phys.\\
        \MakeTextUppercase{Hagen Wierstorf}\\
        geb. in Rotenburg (W\"umme)\\
        }%
        \vspace{1.7cm}%
        \fontsize{18}{20}\selectfont\par\noindent\textcolor{darkgray}{\centering%
        von der Fakult\"at IV -- Elektrotechnik und Informatik\\
        der Technischen Universit\"at Berlin\\
        zur Erlangung des akademischen Grades\\
        }%
        \vspace{0.4cm}
        \fontsize{18}{20}\selectfont\par\noindent\textcolor{darkgray}{\centering%
        Doktor der Naturwissenschaften\\
        -- Dr.~rer.~nat. --\\
        }%
        \vspace{0.4cm}
        \fontsize{18}{20}\selectfont\par\noindent\textcolor{darkgray}{\centering%
        genehmigte Dissertation\\
        }%
        \vspace{1.7cm}%
        \fontsize{18}{20}\selectfont\par\noindent\textcolor{darkgray}{%
        Promotionsausschuss:
        }%
        \vspace{0.5cm}
        \fontsize{18}{20}\selectfont\par\noindent\textcolor{darkgray}{%
        Vorsitzender: Prof.~Dr.-Ing. Sebastian M\"oller\\
        Gutachter:\;\;\;\;\:\,Prof.~Dr.-Ing. Alexander Raake\\
        Gutachter:\;\;\;\;\:\,Prof.~Dr.-Ing. Sascha Spors\\
        Gutachter:\;\;\;\;\:\,Prof.~Dr. Steven van de Par\\
        }%
        \vspace{0.5cm}
        \fontsize{18}{20}\selectfont\par\noindent\textcolor{darkgray}{%
        Tag der wissenschaftlichen Aussprache: 23. September 2014
        }%
        \vspace{0.9cm}
        \fontsize{18}{20}\selectfont\par\noindent\textcolor{darkgray}{\centering%
        Berlin 2014\\
        D~83\\
        \href{https://creativecommons.org/licenses/by/3.0/de/}{%
        CC BY 3.0 DE}\\
        }%
    \end{fullwidth}%
\end{titlepage}

\chapter*{Zusammenfassung}

Die vorliegende Arbeit untersucht die beiden Schall\-feld\-syn\-the\-se\-ver\-fah\-ren
Wellenfeldsynthese und Nahfeld-entzerrtes Am\-bi\-so\-nics hö\-he\-rer Ordnung. Sie fasst
die Theorie der beiden Verfahren zusammen und stellt eine Software-Umgebung zur
Verfügung, um beide Verfahren numerisch zu simulieren.
Diskutiert werden mögliche Abweichungen der mit realen Lautsprechergruppen
synthetisierten Schallfelder. Dies geschieht sowohl auf theoretischer Basis
als auch in einer Reihe von psychoakustischen Experimenten. Die Experimente
untersuchen dabei die räumliche und klangliche Treue und zeitlich-spektrale
Artefakte der verwendeten Systeme. Systematisch wird dies für eine große
Anzahl von verschiedenen Lautsprechergruppen angewendet. Die Experimente werden mit Hilfe
von dynamischer binauraler Synthese durchgeführt, damit auch
Laut\-spre\-cher\-grup\-pen
mit einem Abstand von unter $1$\,cm zwischen den Lautsprechern
untersucht werden können.
Die Ergebnisse zeigen, dass räumliche Treue bereits mit einem
Lautsprecherabstand von $20$\,cm erzielt werden kann, während klangliche Treue
nur mit Abständen kleiner als $1$\,cm möglich ist.
Zeitlich-spektrale Artefakte treten nur bei der Synthese von fo\-kus\-sier\-ten
Quellen auf.
Am Ende wird ein binaurales Modell präsentiert, wel\-ches in der Lage ist die
räumliche Treue für beliebige Lautsprechergruppen vorherzusagen.

\newpage

\chapter*{Abstract}
%
This thesis investigates the two sound field synthesis methods
Wave Field Synthesis and near-field compensated higher order Ambisonics. It
summarizes their theory and provides a software toolkit for corresponding
numerical simulations. Possible deviations of the synthesized sound field for
real loudspeaker arrays and their perceptual relevance are discussed. This is
done firstly based on theoretical considerations, and then addressed in several
psychoacoustic experiments. These experiments investigate the spatial and
timbral fidelity and spectro-temporal-artifacts in a systematic way for a large
number of different loudspeaker setups. The experiments are conducted with the
help of dynamic binaural synthesis in order to simulate loudspeaker setups with
an inter-loudspeaker spacing of under 1\,cm.
The results show that spatial fidelity can already be achieved with setups
having an inter-loudspeaker spacing of $20$\,cm,
whereas timbral fidelity is only possible for setups employing a spacing below
$1$\,cm.
Spectro-temporal artifacts are relevant only for the synthesis of focused
sources.
At the end of the thesis, a binaural auditory model is presented that is able
to predict the spatial fidelity for any given loudspeaker setup.


\newpage
