\chapter{Conclusions}
\label{cha:conclusions}

\firstthought{Sound field synthesis} allows to create controllable sound fields
in an extended volume. This emphasizes the usage of such methods in spatial
sound presentation in order to convey a rich audio experience to the
listener. The shortcoming of \ac{SFS} is its underlying theoretical assumption
of continuous loudspeaker distributions that are not possible to build in
practice. 
Nonetheless, experiences with build sound field synthesis setups showed that
a spatial convincing presentation is nonetheless possible, by accepting
slight degradations of the perceived timbre.

The goal of this thesis was to investigate in more detail the employment
of sound field synthesis for spatial sound presentations.
In the introduction a couple of
corresponding research questions were formulated that are repeated here
and tried to answered in the light of the achieved results.
\emph{What is the best way to give a good spatial
impression of the presented sound? How many loudspeakers are needed to do this?
Is it possible with the current hardware limitations to create a whole sound
field in a convincing way? What is the influence of the spatial impression on
the overall quality a listener experiences while listening to the played back
sound?}

The results from the spatial fidelity experiments in
Section\,\ref{sec:localization} indicate that especially
\ac{WFS} gives a good spatial impression in the whole audience area.
This was achieved even for the relatively low number of 28 loudspeakers for a
circular loudspeaker array with a diameter of $3$\,m and a corresponding
distance between them of $34$\,cm. Further \ac{WFS} can be
implemented in an efficient way and arbitrary geometries of
loudspeaker arrays can be used.

Considering not only the spatial but the overall impression of the synthesized
sound field, results become more subtle. Rumsey et al.\autocite{Rumsey2005}
conducted an experiment in which listeners rated the spatial fidelity, timbral
fidelity, and the perceived overall quality of stereophonic surround setups. The
results show that the contribution of spatial fidelity to the overall quality
is only $30\%$. Timbral fidelity accounts for the rest. Their findings
demonstrate that for \ac{SFS} timbral fidelity might have the same importance for
the perceived quality.
The results from the experiments in
Section\,\ref{sec:timbral_fidelity} show that timbral fidelity could only be
achieved by employing more than 3\,000 loudspeakers for a
circular loudspeaker array with a diameter of $3$\,m. However, practical setups
with the same size will seldom have more than 64 loudspeakers. In this case
timbral fidelity cannot be achieved. On the other hand the same is true for
two-channel stereophony, without large impairments of the perceived quality.

Further investigations of sound field synthesis in comparison to stereophonic techniques
should be carried out to assess its perceived quality.
Therefore, more complex sound scenes should be synthesized and listener should
rate the plausibility of the corresponding auditory scene.
Another research topic is of interest in order to understand the influence of
sound field errors on its perceived quality. The experimental results from
this thesis point to a close connection between the perception of coloration and
the precedence effect and one hypothesis is that both are part of the same
mechanism in the brain.

By limiting the audience area to a region of approximately the size of a human
head, band-limited \ac{NFC-HOA} is an interesting approach that can provide a
convincing sound field.
The limitation to a small region enables the synthesis of an error free sound
field which achieves spatial and timbral fidelity in this region.


%%%%%%%%%%%%%%%%%%%%%%%%%%%%%%%%%%%%%%%%%%%%%%%%%%%%%%%%%%%%%%%%%%%%%%%%%%%%%%%%

% Final stuff

\newpage

\newpage

\setcounter{footnote}{0}

\section*{Further Resources}
%
This section is devoted to the acknowledgements of further literature that had
an impact on the writing of this thesis but was not mentioned so far.

The underlying design principals of the figures in this thesis are motivated by
Tuftes work.\sidenote[][-1.5cm]{\cite[E.g.][]{Tufte2011}}
The colormap used for plotting all of the numerical sound field simulations is
given by Moreland.\sidenote[][-0.8cm]{\cite{Moreland2009}} Most of the colors in the other
figures are based on colormaps published by
Brewer.\sidenote{\url{colorbrewer2.org}{http://colorbrewer2.org}\newline\cite{Brewer2005}}

The idea of predicting the localization in the whole audience area has its
origin in a paper by Merchel et al.\autocite{Merchel2010}


